%% Generated by Sphinx.
\def\sphinxdocclass{report}
\documentclass[letterpaper,10pt,english]{sphinxmanual}
\ifdefined\pdfpxdimen
   \let\sphinxpxdimen\pdfpxdimen\else\newdimen\sphinxpxdimen
\fi \sphinxpxdimen=.75bp\relax
\ifdefined\pdfimageresolution
    \pdfimageresolution= \numexpr \dimexpr1in\relax/\sphinxpxdimen\relax
\fi
%% let collapsable pdf bookmarks panel have high depth per default
\PassOptionsToPackage{bookmarksdepth=5}{hyperref}

\PassOptionsToPackage{warn}{textcomp}
\usepackage[utf8]{inputenc}
\ifdefined\DeclareUnicodeCharacter
% support both utf8 and utf8x syntaxes
  \ifdefined\DeclareUnicodeCharacterAsOptional
    \def\sphinxDUC#1{\DeclareUnicodeCharacter{"#1}}
  \else
    \let\sphinxDUC\DeclareUnicodeCharacter
  \fi
  \sphinxDUC{00A0}{\nobreakspace}
  \sphinxDUC{2500}{\sphinxunichar{2500}}
  \sphinxDUC{2502}{\sphinxunichar{2502}}
  \sphinxDUC{2514}{\sphinxunichar{2514}}
  \sphinxDUC{251C}{\sphinxunichar{251C}}
  \sphinxDUC{2572}{\textbackslash}
\fi
\usepackage{cmap}
\usepackage[T1]{fontenc}
\usepackage{amsmath,amssymb,amstext}
\usepackage{babel}



\usepackage{tgtermes}
\usepackage{tgheros}
\renewcommand{\ttdefault}{txtt}



\usepackage[Bjarne]{fncychap}
\usepackage{sphinx}

\fvset{fontsize=auto}
\usepackage{geometry}


% Include hyperref last.
\usepackage{hyperref}
% Fix anchor placement for figures with captions.
\usepackage{hypcap}% it must be loaded after hyperref.
% Set up styles of URL: it should be placed after hyperref.
\urlstyle{same}

\addto\captionsenglish{\renewcommand{\contentsname}{Usage}}

\usepackage{sphinxmessages}
\setcounter{tocdepth}{1}



\title{dtwhaclustering}
\date{Aug 20, 2021}
\release{1.0}
\author{Utpal Kumar}
\newcommand{\sphinxlogo}{\vbox{}}
\renewcommand{\releasename}{Release}
\makeindex
\begin{document}

\pagestyle{empty}
\sphinxmaketitle
\pagestyle{plain}
\sphinxtableofcontents
\pagestyle{normal}
\phantomsection\label{\detokenize{index::doc}}



\chapter{Installation}
\label{\detokenize{usage/install:installation}}\label{\detokenize{usage/install::doc}}

\section{Requirements}
\label{\detokenize{usage/install:requirements}}\begin{enumerate}
\sphinxsetlistlabels{\arabic}{enumi}{enumii}{}{.}%
\item {} 
\sphinxAtStartPar
\sphinxcode{\sphinxupquote{dtaidistance}}: For computing DTW distance

\item {} 
\sphinxAtStartPar
\sphinxcode{\sphinxupquote{pygmt}}: For plotting high\sphinxhyphen{}resolution maps

\item {} 
\sphinxAtStartPar
\sphinxcode{\sphinxupquote{pandas}}: Analyze tabular data

\item {} 
\sphinxAtStartPar
\sphinxcode{\sphinxupquote{numpy}}: Computation

\item {} 
\sphinxAtStartPar
\sphinxcode{\sphinxupquote{matplotlib}}: Plotting time series

\item {} 
\sphinxAtStartPar
\sphinxcode{\sphinxupquote{scipy}}: Interpolating data

\item {} 
\sphinxAtStartPar
\sphinxcode{\sphinxupquote{xarray}}: Multilayered data structure

\end{enumerate}


\section{Install from PyPI}
\label{\detokenize{usage/install:install-from-pypi}}
\sphinxAtStartPar
This package is available on PyPI (requires Python 3):

\begin{sphinxVerbatim}[commandchars=\\\{\}]
\PYG{n}{pip} \PYG{n}{install} \PYG{n}{dtwhaclustering}
\end{sphinxVerbatim}
\phantomsection\label{\detokenize{modules/module_contents:module-dtwhaclustering}}\index{module@\spxentry{module}!dtwhaclustering@\spxentry{dtwhaclustering}}\index{dtwhaclustering@\spxentry{dtwhaclustering}!module@\spxentry{module}}

\chapter{Dynamic Time Warping based Hierarchical Agglomerative Clustering}
\label{\detokenize{modules/module_contents:dynamic-time-warping-based-hierarchical-agglomerative-clustering}}\label{\detokenize{modules/module_contents::doc}}
\sphinxAtStartPar
Codes to perform Dynamic Time Warping Based Hierarchical Agglomerative Clustering of GPS data
\begin{quote}\begin{description}
\item[{author}] \leavevmode
\sphinxAtStartPar
Utpal Kumar

\item[{date}] \leavevmode
\sphinxAtStartPar
2021/08

\item[{copyright}] \leavevmode
\sphinxAtStartPar
2021, Institute of Earth Sciences, Academia Sinica.

\end{description}\end{quote}


\chapter{dtwhaclustering.analysis\_support}
\label{\detokenize{modules/analysis_support:module-dtwhaclustering.analysis_support}}\label{\detokenize{modules/analysis_support:dtwhaclustering-analysis-support}}\label{\detokenize{modules/analysis_support::doc}}\index{module@\spxentry{module}!dtwhaclustering.analysis\_support@\spxentry{dtwhaclustering.analysis\_support}}\index{dtwhaclustering.analysis\_support@\spxentry{dtwhaclustering.analysis\_support}!module@\spxentry{module}}
\sphinxAtStartPar
DTW HAC analysis support functions (\sphinxtitleref{analysis\_support})
\begin{quote}\begin{description}
\item[{author}] \leavevmode
\sphinxAtStartPar
Utpal Kumar, Institute of Earth Sciences, Academia Sinica

\end{description}\end{quote}
\index{dec2dt() (in module dtwhaclustering.analysis\_support)@\spxentry{dec2dt()}\spxextra{in module dtwhaclustering.analysis\_support}}

\begin{fulllineitems}
\phantomsection\label{\detokenize{modules/analysis_support:dtwhaclustering.analysis_support.dec2dt}}\pysiglinewithargsret{\sphinxcode{\sphinxupquote{dtwhaclustering.analysis\_support.}}\sphinxbfcode{\sphinxupquote{dec2dt}}}{\emph{\DUrole{n}{start}}}{}
\sphinxAtStartPar
Convert the decimal type time array to the date\sphinxhyphen{}time type array
\begin{quote}\begin{description}
\item[{Parameters}] \leavevmode
\sphinxAtStartPar
\sphinxstyleliteralstrong{\sphinxupquote{start}} (\sphinxstyleliteralemphasis{\sphinxupquote{list}}) \textendash{} list or numpy array of decimal year values e.g., {[}2020.001{]}

\item[{Returns}] \leavevmode
\sphinxAtStartPar
date\sphinxhyphen{}time type array

\item[{Return type}] \leavevmode
\sphinxAtStartPar
list

\end{description}\end{quote}

\end{fulllineitems}

\index{dec2dt\_scalar() (in module dtwhaclustering.analysis\_support)@\spxentry{dec2dt\_scalar()}\spxextra{in module dtwhaclustering.analysis\_support}}

\begin{fulllineitems}
\phantomsection\label{\detokenize{modules/analysis_support:dtwhaclustering.analysis_support.dec2dt_scalar}}\pysiglinewithargsret{\sphinxcode{\sphinxupquote{dtwhaclustering.analysis\_support.}}\sphinxbfcode{\sphinxupquote{dec2dt\_scalar}}}{\emph{\DUrole{n}{st}}}{}
\sphinxAtStartPar
Convert the decimal type time value to the date\sphinxhyphen{}time type
\begin{quote}\begin{description}
\item[{Parameters}] \leavevmode
\sphinxAtStartPar
\sphinxstyleliteralstrong{\sphinxupquote{st}} \textendash{} scalar decimal year value e.g., 2020.001

\item[{Returns}] \leavevmode
\sphinxAtStartPar
time as datetime type

\item[{Return type}] \leavevmode
\sphinxAtStartPar
str

\end{description}\end{quote}

\end{fulllineitems}

\index{toYearFraction() (in module dtwhaclustering.analysis\_support)@\spxentry{toYearFraction()}\spxextra{in module dtwhaclustering.analysis\_support}}

\begin{fulllineitems}
\phantomsection\label{\detokenize{modules/analysis_support:dtwhaclustering.analysis_support.toYearFraction}}\pysiglinewithargsret{\sphinxcode{\sphinxupquote{dtwhaclustering.analysis\_support.}}\sphinxbfcode{\sphinxupquote{toYearFraction}}}{\emph{\DUrole{n}{date}}}{}
\sphinxAtStartPar
Convert the date\sphinxhyphen{}time type object to decimal year
\begin{quote}\begin{description}
\item[{Parameters}] \leavevmode
\sphinxAtStartPar
\sphinxstyleliteralstrong{\sphinxupquote{date}} \textendash{} the date\sphinxhyphen{}time type object

\item[{Returns}] \leavevmode
\sphinxAtStartPar
decimal year

\item[{Return type}] \leavevmode
\sphinxAtStartPar
float

\end{description}\end{quote}

\end{fulllineitems}



\chapter{dtwhaclustering.plot\_linear\_trend}
\label{\detokenize{modules/plot_linear_trend:module-dtwhaclustering.plot_linear_trend}}\label{\detokenize{modules/plot_linear_trend:dtwhaclustering-plot-linear-trend}}\label{\detokenize{modules/plot_linear_trend::doc}}\index{module@\spxentry{module}!dtwhaclustering.plot\_linear\_trend@\spxentry{dtwhaclustering.plot\_linear\_trend}}\index{dtwhaclustering.plot\_linear\_trend@\spxentry{dtwhaclustering.plot\_linear\_trend}!module@\spxentry{module}}
\sphinxAtStartPar
Plot linear trend values on a geographical map
\begin{quote}\begin{description}
\item[{author}] \leavevmode
\sphinxAtStartPar
Utpal Kumar, Institute of Earth Sciences, Academia Sinica

\end{description}\end{quote}
\index{compute\_interpolation() (in module dtwhaclustering.plot\_linear\_trend)@\spxentry{compute\_interpolation()}\spxextra{in module dtwhaclustering.plot\_linear\_trend}}

\begin{fulllineitems}
\phantomsection\label{\detokenize{modules/plot_linear_trend:dtwhaclustering.plot_linear_trend.compute_interpolation}}\pysiglinewithargsret{\sphinxcode{\sphinxupquote{dtwhaclustering.plot\_linear\_trend.}}\sphinxbfcode{\sphinxupquote{compute\_interpolation}}}{\emph{\DUrole{n}{df}}, \emph{\DUrole{n}{method}\DUrole{o}{=}\DUrole{default_value}{\textquotesingle{}nearest\textquotesingle{}}}, \emph{\DUrole{n}{lonrange}\DUrole{o}{=}\DUrole{default_value}{(120.0, 122.0)}}, \emph{\DUrole{n}{latrange}\DUrole{o}{=}\DUrole{default_value}{(21.8, 25.6)}}, \emph{\DUrole{n}{step}\DUrole{o}{=}\DUrole{default_value}{0.01}}}{}
\sphinxAtStartPar
Interpolate linear trend values using Scipy’s griddata and returns xarray object
\begin{quote}\begin{description}
\item[{Parameters}] \leavevmode\begin{itemize}
\item {} 
\sphinxAtStartPar
\sphinxstyleliteralstrong{\sphinxupquote{df}} (\sphinxstyleliteralemphasis{\sphinxupquote{pandas.DataFrame}}) \textendash{} pandas dataframe containing the columns \sphinxtitleref{lon}, \sphinxtitleref{lat} and \sphinxtitleref{slope}

\item {} 
\sphinxAtStartPar
\sphinxstyleliteralstrong{\sphinxupquote{method}} (\sphinxstyleliteralemphasis{\sphinxupquote{str}}) \textendash{} Method of interpolation. One of \{‘linear’, ‘nearest’, ‘cubic’\}. For more, see \sphinxtitleref{scipy.interpolate.griddata}

\item {} 
\sphinxAtStartPar
\sphinxstyleliteralstrong{\sphinxupquote{lonrange}} (\sphinxstyleliteralemphasis{\sphinxupquote{tuple}}) \textendash{} minimum and maximum values of the longitude to be interpolated

\item {} 
\sphinxAtStartPar
\sphinxstyleliteralstrong{\sphinxupquote{latrange}} (\sphinxstyleliteralemphasis{\sphinxupquote{tuple}}) \textendash{} minimum and maximum values of the latitude to be interpolated

\item {} 
\sphinxAtStartPar
\sphinxstyleliteralstrong{\sphinxupquote{step}} (\sphinxstyleliteralemphasis{\sphinxupquote{float}}) \textendash{} stepsize to interpolate data spatially

\end{itemize}

\item[{Returns}] \leavevmode
\sphinxAtStartPar
\sphinxtitleref{xarray.DataArray} of dims, and coords (“lat”, “long”), maximum of the absolute interpolated array values

\end{description}\end{quote}

\end{fulllineitems}

\index{plot\_linear\_trend\_on\_map() (in module dtwhaclustering.plot\_linear\_trend)@\spxentry{plot\_linear\_trend\_on\_map()}\spxextra{in module dtwhaclustering.plot\_linear\_trend}}

\begin{fulllineitems}
\phantomsection\label{\detokenize{modules/plot_linear_trend:dtwhaclustering.plot_linear_trend.plot_linear_trend_on_map}}\pysiglinewithargsret{\sphinxcode{\sphinxupquote{dtwhaclustering.plot\_linear\_trend.}}\sphinxbfcode{\sphinxupquote{plot\_linear\_trend\_on\_map}}}{\emph{\DUrole{n}{df}}, \emph{\DUrole{n}{maplonrange}\DUrole{o}{=}\DUrole{default_value}{(120.0, 122.1)}}, \emph{\DUrole{n}{maplatrange}\DUrole{o}{=}\DUrole{default_value}{(21.8, 25.6)}}, \emph{\DUrole{n}{intrp\_lonrange}\DUrole{o}{=}\DUrole{default_value}{(120.0, 122.0)}}, \emph{\DUrole{n}{intrp\_latrange}\DUrole{o}{=}\DUrole{default_value}{(21.8, 25.6)}}, \emph{\DUrole{n}{outfig}\DUrole{o}{=}\DUrole{default_value}{\textquotesingle{}Maps/slope\sphinxhyphen{}plot.png\textquotesingle{}}}, \emph{\DUrole{n}{frame}\DUrole{o}{=}\DUrole{default_value}{{[}\textquotesingle{}a1f0.25\textquotesingle{}, \textquotesingle{}WSen\textquotesingle{}{]}}}, \emph{\DUrole{n}{cmap}\DUrole{o}{=}\DUrole{default_value}{\textquotesingle{}jet\textquotesingle{}}}, \emph{\DUrole{n}{step}\DUrole{o}{=}\DUrole{default_value}{0.01}}, \emph{\DUrole{n}{stn\_labels}\DUrole{o}{=}\DUrole{default_value}{None}}, \emph{\DUrole{n}{justify}\DUrole{o}{=}\DUrole{default_value}{\textquotesingle{}left\textquotesingle{}}}, \emph{\DUrole{n}{labelfont}\DUrole{o}{=}\DUrole{default_value}{\textquotesingle{}6p,Helvetica\sphinxhyphen{}Bold,black\textquotesingle{}}}, \emph{\DUrole{n}{offset}\DUrole{o}{=}\DUrole{default_value}{\textquotesingle{}5p/\sphinxhyphen{}5p\textquotesingle{}}}, \emph{\DUrole{n}{markerstyle}\DUrole{o}{=}\DUrole{default_value}{\textquotesingle{}i10p\textquotesingle{}}}, \emph{\DUrole{n}{defpen}\DUrole{o}{=}\DUrole{default_value}{\textquotesingle{}1p,black\textquotesingle{}}}, \emph{\DUrole{n}{stn\_labels\_color}\DUrole{o}{=}\DUrole{default_value}{\textquotesingle{}black\textquotesingle{}}}, \emph{\DUrole{n}{rand\_justify}\DUrole{o}{=}\DUrole{default_value}{False}}, \emph{\DUrole{n}{water\_color}\DUrole{o}{=}\DUrole{default_value}{\textquotesingle{}skyblue\textquotesingle{}}}}{}
\sphinxAtStartPar
Plot the interpolated linear trend values along with the original data points on a geographical map using PyGMT
\begin{quote}\begin{description}
\item[{Parameters}] \leavevmode\begin{itemize}
\item {} 
\sphinxAtStartPar
\sphinxstyleliteralstrong{\sphinxupquote{df}} (\sphinxstyleliteralemphasis{\sphinxupquote{pandas.DataFrame}}) \textendash{} Pandas dataframe containing the columns \sphinxtitleref{lon}, \sphinxtitleref{lat} and \sphinxtitleref{slope}

\item {} 
\sphinxAtStartPar
\sphinxstyleliteralstrong{\sphinxupquote{maplonrange}} (\sphinxstyleliteralemphasis{\sphinxupquote{tuple}}) \textendash{} longitude min/max of the output map

\item {} 
\sphinxAtStartPar
\sphinxstyleliteralstrong{\sphinxupquote{maplatrange}} (\sphinxstyleliteralemphasis{\sphinxupquote{tuple}}) \textendash{} latitude min/max of the output map

\item {} 
\sphinxAtStartPar
\sphinxstyleliteralstrong{\sphinxupquote{intrp\_lonrange}} (\sphinxstyleliteralemphasis{\sphinxupquote{tuple}}) \textendash{} longitude min/max for the interpolation of data

\item {} 
\sphinxAtStartPar
\sphinxstyleliteralstrong{\sphinxupquote{intrp\_latrange}} (\sphinxstyleliteralemphasis{\sphinxupquote{tuple}}) \textendash{} latitude min/max for the interpolation of data

\item {} 
\sphinxAtStartPar
\sphinxstyleliteralstrong{\sphinxupquote{step}} (\sphinxstyleliteralemphasis{\sphinxupquote{float}}) \textendash{} resolution of the interpolation

\item {} 
\sphinxAtStartPar
\sphinxstyleliteralstrong{\sphinxupquote{cmap}} (\sphinxstyleliteralemphasis{\sphinxupquote{str}}) \textendash{} colormap for the output map

\item {} 
\sphinxAtStartPar
\sphinxstyleliteralstrong{\sphinxupquote{frame}} (\sphinxstyleliteralemphasis{\sphinxupquote{list}}) \textendash{} frame of the output map. See PyGMT docs for details

\item {} 
\sphinxAtStartPar
\sphinxstyleliteralstrong{\sphinxupquote{outfig}} (\sphinxstyleliteralemphasis{\sphinxupquote{str}}) \textendash{} output figure name with extension, e.g., \sphinxtitleref{slope\sphinxhyphen{}plot.png}

\item {} 
\sphinxAtStartPar
\sphinxstyleliteralstrong{\sphinxupquote{water\_color}} (\sphinxstyleliteralemphasis{\sphinxupquote{str}}) \textendash{} color of the water, default: skyblue

\item {} 
\sphinxAtStartPar
\sphinxstyleliteralstrong{\sphinxupquote{stn\_labels}} (\sphinxstyleliteralemphasis{\sphinxupquote{array}}) \textendash{} label the station names. Only the stations provided will be labeled

\item {} 
\sphinxAtStartPar
\sphinxstyleliteralstrong{\sphinxupquote{justify}} (\sphinxstyleliteralemphasis{\sphinxupquote{str}}) \textendash{} justification of the station labels \sphinxhyphen{} ‘left’, ‘right’

\item {} 
\sphinxAtStartPar
\sphinxstyleliteralstrong{\sphinxupquote{rand\_justify}} (\sphinxstyleliteralemphasis{\sphinxupquote{boolean}}) \textendash{} randomly decide the justification for each labels

\item {} 
\sphinxAtStartPar
\sphinxstyleliteralstrong{\sphinxupquote{markerstyle}} (\sphinxstyleliteralemphasis{\sphinxupquote{str}}) \textendash{} marker style and size

\end{itemize}

\item[{Returns}] \leavevmode
\sphinxAtStartPar
None

\end{description}\end{quote}

\end{fulllineitems}



\chapter{dtwhaclustering.leastSquareModeling}
\label{\detokenize{modules/leastSquareModeling:module-dtwhaclustering.leastSquareModeling}}\label{\detokenize{modules/leastSquareModeling:dtwhaclustering-leastsquaremodeling}}\label{\detokenize{modules/leastSquareModeling::doc}}\index{module@\spxentry{module}!dtwhaclustering.leastSquareModeling@\spxentry{dtwhaclustering.leastSquareModeling}}\index{dtwhaclustering.leastSquareModeling@\spxentry{dtwhaclustering.leastSquareModeling}!module@\spxentry{module}}
\sphinxAtStartPar
Least square modeling of GPS displacements for seasonality, tidal, co\sphinxhyphen{}seismic jumps.
\begin{quote}\begin{description}
\item[{author}] \leavevmode
\sphinxAtStartPar
Utpal Kumar, Institute of Earth Sciences, Academia Sinica

\end{description}\end{quote}
\index{LSQmodules (class in dtwhaclustering.leastSquareModeling)@\spxentry{LSQmodules}\spxextra{class in dtwhaclustering.leastSquareModeling}}

\begin{fulllineitems}
\phantomsection\label{\detokenize{modules/leastSquareModeling:dtwhaclustering.leastSquareModeling.LSQmodules}}\pysiglinewithargsret{\sphinxbfcode{\sphinxupquote{class }}\sphinxcode{\sphinxupquote{dtwhaclustering.leastSquareModeling.}}\sphinxbfcode{\sphinxupquote{LSQmodules}}}{\emph{\DUrole{n}{dUU}}, \emph{\DUrole{n}{sel\_eq\_file}\DUrole{o}{=}\DUrole{default_value}{None}}, \emph{\DUrole{n}{station\_loc\_file}\DUrole{o}{=}\DUrole{default_value}{\textquotesingle{}helper\_files/stn\_loc.txt\textquotesingle{}}}, \emph{\DUrole{n}{comp}\DUrole{o}{=}\DUrole{default_value}{\textquotesingle{}U\textquotesingle{}}}, \emph{\DUrole{n}{figdir}\DUrole{o}{=}\DUrole{default_value}{\textquotesingle{}LSQOut\textquotesingle{}}}, \emph{\DUrole{n}{periods}\DUrole{o}{=}\DUrole{default_value}{(13.6608, 14.7653, 27.5546, 182.62, 365.26, 6793.836)}}}{}
\sphinxAtStartPar
Bases: \sphinxcode{\sphinxupquote{object}}
\index{compute\_lsq() (dtwhaclustering.leastSquareModeling.LSQmodules method)@\spxentry{compute\_lsq()}\spxextra{dtwhaclustering.leastSquareModeling.LSQmodules method}}

\begin{fulllineitems}
\phantomsection\label{\detokenize{modules/leastSquareModeling:dtwhaclustering.leastSquareModeling.LSQmodules.compute_lsq}}\pysiglinewithargsret{\sphinxbfcode{\sphinxupquote{compute\_lsq}}}{\emph{\DUrole{n}{plot\_results}\DUrole{o}{=}\DUrole{default_value}{False}}, \emph{\DUrole{n}{remove\_trend}\DUrole{o}{=}\DUrole{default_value}{True}}, \emph{\DUrole{n}{remove\_seasonality}\DUrole{o}{=}\DUrole{default_value}{True}}, \emph{\DUrole{n}{remove\_jumps}\DUrole{o}{=}\DUrole{default_value}{True}}, \emph{\DUrole{n}{plotformat}\DUrole{o}{=}\DUrole{default_value}{None}}}{}
\sphinxAtStartPar
Compute the least\sphinxhyphen{}squares model using multithreading
\begin{quote}\begin{description}
\item[{Parameters}] \leavevmode\begin{itemize}
\item {} 
\sphinxAtStartPar
\sphinxstyleliteralstrong{\sphinxupquote{plot\_results}} (\sphinxstyleliteralemphasis{\sphinxupquote{boolean}}) \textendash{} plot the final results

\item {} 
\sphinxAtStartPar
\sphinxstyleliteralstrong{\sphinxupquote{remove\_trend}} (\sphinxstyleliteralemphasis{\sphinxupquote{boolean}}) \textendash{} return the time series after removing the linear trend

\item {} 
\sphinxAtStartPar
\sphinxstyleliteralstrong{\sphinxupquote{remove\_seasonality}} (\sphinxstyleliteralemphasis{\sphinxupquote{boolean}}) \textendash{} return the time series after removing the seasonal signals

\item {} 
\sphinxAtStartPar
\sphinxstyleliteralstrong{\sphinxupquote{remove\_jumps}} (\sphinxstyleliteralemphasis{\sphinxupquote{boolean}}) \textendash{} return the time series after removing the co\sphinxhyphen{}seismic jumps

\item {} 
\sphinxAtStartPar
\sphinxstyleliteralstrong{\sphinxupquote{plotformat}} (\sphinxstyleliteralemphasis{\sphinxupquote{str}}) \textendash{} plot format of the output figure, e.g. “png”. “pdf” by default.

\end{itemize}

\end{description}\end{quote}

\end{fulllineitems}

\index{jump() (dtwhaclustering.leastSquareModeling.LSQmodules method)@\spxentry{jump()}\spxextra{dtwhaclustering.leastSquareModeling.LSQmodules method}}

\begin{fulllineitems}
\phantomsection\label{\detokenize{modules/leastSquareModeling:dtwhaclustering.leastSquareModeling.LSQmodules.jump}}\pysiglinewithargsret{\sphinxbfcode{\sphinxupquote{jump}}}{\emph{\DUrole{n}{t}}, \emph{\DUrole{n}{t0}}}{}
\sphinxAtStartPar
heaviside step function
\begin{quote}\begin{description}
\item[{Parameters}] \leavevmode\begin{itemize}
\item {} 
\sphinxAtStartPar
\sphinxstyleliteralstrong{\sphinxupquote{t}} (\sphinxstyleliteralemphasis{\sphinxupquote{list}}) \textendash{} time data

\item {} 
\sphinxAtStartPar
\sphinxstyleliteralstrong{\sphinxupquote{t0}} \textendash{} earthquake origin time

\end{itemize}

\end{description}\end{quote}

\end{fulllineitems}


\end{fulllineitems}

\index{lsqmodeling() (in module dtwhaclustering.leastSquareModeling)@\spxentry{lsqmodeling()}\spxextra{in module dtwhaclustering.leastSquareModeling}}

\begin{fulllineitems}
\phantomsection\label{\detokenize{modules/leastSquareModeling:dtwhaclustering.leastSquareModeling.lsqmodeling}}\pysiglinewithargsret{\sphinxcode{\sphinxupquote{dtwhaclustering.leastSquareModeling.}}\sphinxbfcode{\sphinxupquote{lsqmodeling}}}{\emph{\DUrole{n}{dUU}}, \emph{\DUrole{n}{dNN}}, \emph{\DUrole{n}{dEE}}, \emph{\DUrole{n}{stnlocfile}}, \emph{\DUrole{n}{plot\_results}\DUrole{o}{=}\DUrole{default_value}{True}}, \emph{\DUrole{n}{remove\_trend}\DUrole{o}{=}\DUrole{default_value}{False}}, \emph{\DUrole{n}{remove\_seasonality}\DUrole{o}{=}\DUrole{default_value}{True}}, \emph{\DUrole{n}{remove\_jumps}\DUrole{o}{=}\DUrole{default_value}{False}}, \emph{\DUrole{n}{sel\_eq\_file}\DUrole{o}{=}\DUrole{default_value}{\textquotesingle{}helper\_files/selected\_eqs\_new.txt\textquotesingle{}}}, \emph{\DUrole{n}{figdir}\DUrole{o}{=}\DUrole{default_value}{\textquotesingle{}LSQOut\textquotesingle{}}}}{}
\sphinxAtStartPar
Least square modeling for the three component time series
\begin{quote}\begin{description}
\item[{Parameters}] \leavevmode\begin{itemize}
\item {} 
\sphinxAtStartPar
\sphinxstyleliteralstrong{\sphinxupquote{dUU}} (\sphinxstyleliteralemphasis{\sphinxupquote{pandas.DataFrame}}) \textendash{} Vertical component pandas dataframe time series

\item {} 
\sphinxAtStartPar
\sphinxstyleliteralstrong{\sphinxupquote{dNN}} (\sphinxstyleliteralemphasis{\sphinxupquote{pandas.DataFrame}}) \textendash{} North component pandas dataframe time series

\item {} 
\sphinxAtStartPar
\sphinxstyleliteralstrong{\sphinxupquote{dEE}} (\sphinxstyleliteralemphasis{\sphinxupquote{pandas.DataFrame}}) \textendash{} East component pandas dataframe time series

\item {} 
\sphinxAtStartPar
\sphinxstyleliteralstrong{\sphinxupquote{plot\_results}} (\sphinxstyleliteralemphasis{\sphinxupquote{boolean}}) \textendash{} plot the final results

\item {} 
\sphinxAtStartPar
\sphinxstyleliteralstrong{\sphinxupquote{remove\_trend}} (\sphinxstyleliteralemphasis{\sphinxupquote{boolean}}) \textendash{} return the time series after removing the linear trend

\item {} 
\sphinxAtStartPar
\sphinxstyleliteralstrong{\sphinxupquote{remove\_seasonality}} (\sphinxstyleliteralemphasis{\sphinxupquote{boolean}}) \textendash{} return the time series after removing the seasonal signals

\item {} 
\sphinxAtStartPar
\sphinxstyleliteralstrong{\sphinxupquote{remove\_jumps}} (\sphinxstyleliteralemphasis{\sphinxupquote{boolean}}) \textendash{} return the time series after removing the co\sphinxhyphen{}seismic jumps

\item {} 
\sphinxAtStartPar
\sphinxstyleliteralstrong{\sphinxupquote{sel\_eq\_file}} (\sphinxstyleliteralemphasis{\sphinxupquote{str}}) \textendash{} File containing the earthquake origin times e.g., \sphinxtitleref{2009,1,15} with header info, e.g. \sphinxtitleref{year\_val,month\_val,date\_val}

\item {} 
\sphinxAtStartPar
\sphinxstyleliteralstrong{\sphinxupquote{stnlocfile}} (\sphinxstyleliteralemphasis{\sphinxupquote{str}}) \textendash{} File containing the station location info, e.g., \sphinxtitleref{DAWU,120.89004,22.34059}, with header \sphinxtitleref{stn,lon,lat}

\end{itemize}

\item[{Returns}] \leavevmode
\sphinxAtStartPar
Pandas dataframe corresponding to the vertical, north and east components e.g., final\_dU, final\_dN, final\_dE

\item[{Return type}] \leavevmode
\sphinxAtStartPar
pandas.DataFrame, pandas.DataFrame, pandas.DataFrame

\end{description}\end{quote}

\begin{sphinxVerbatim}[commandchars=\\\{\}]
\PYG{k+kn}{from} \PYG{n+nn}{dtwhaclustering}\PYG{n+nn}{.}\PYG{n+nn}{leastSquareModeling} \PYG{k+kn}{import} \PYG{n}{lsqmodeling}
\PYG{n}{final\PYGZus{}dU}\PYG{p}{,} \PYG{n}{final\PYGZus{}dN}\PYG{p}{,} \PYG{n}{final\PYGZus{}dE} \PYG{o}{=} \PYG{n}{lsqmodeling}\PYG{p}{(}\PYG{n}{dUU}\PYG{p}{,} \PYG{n}{dNN}\PYG{p}{,} \PYG{n}{dEE}\PYG{p}{,}\PYG{n}{stnlocfile}\PYG{o}{=}\PYG{l+s+s2}{\PYGZdq{}}\PYG{l+s+s2}{helper\PYGZus{}files/stn\PYGZus{}loc.txt}\PYG{l+s+s2}{\PYGZdq{}}\PYG{p}{,}  \PYG{n}{plot\PYGZus{}results}\PYG{o}{=}\PYG{k+kc}{True}\PYG{p}{,} \PYG{n}{remove\PYGZus{}trend}\PYG{o}{=}\PYG{k+kc}{False}\PYG{p}{,} \PYG{n}{remove\PYGZus{}seasonality}\PYG{o}{=}\PYG{k+kc}{True}\PYG{p}{,} \PYG{n}{remove\PYGZus{}jumps}\PYG{o}{=}\PYG{k+kc}{False}\PYG{p}{)}
\end{sphinxVerbatim}

\end{fulllineitems}



\chapter{dtwhaclustering.dtw\_analysis}
\label{\detokenize{modules/dtw_analysis:module-dtwhaclustering.dtw_analysis}}\label{\detokenize{modules/dtw_analysis:dtwhaclustering-dtw-analysis}}\label{\detokenize{modules/dtw_analysis::doc}}\index{module@\spxentry{module}!dtwhaclustering.dtw\_analysis@\spxentry{dtwhaclustering.dtw\_analysis}}\index{dtwhaclustering.dtw\_analysis@\spxentry{dtwhaclustering.dtw\_analysis}!module@\spxentry{module}}
\sphinxAtStartPar
Classes and functions for the DTW analysis and plotting maps and figures (\sphinxtitleref{dtw\_analysis})
This module is built around the dtaidistance package for the DTW computation and scipy.cluster
\begin{quote}\begin{description}
\item[{author}] \leavevmode
\sphinxAtStartPar
Utpal Kumar, Institute of Earth Sciences, Academia Sinica

\item[{note}] \leavevmode
\sphinxAtStartPar
See \sphinxhref{https://dtaidistance.readthedocs.io/en/latest/usage/dtw.html}{dtaidistance} for details on HierarchicalTree, dtw, dtw\_visualisation

\end{description}\end{quote}
\index{dtw\_clustering (class in dtwhaclustering.dtw\_analysis)@\spxentry{dtw\_clustering}\spxextra{class in dtwhaclustering.dtw\_analysis}}

\begin{fulllineitems}
\phantomsection\label{\detokenize{modules/dtw_analysis:dtwhaclustering.dtw_analysis.dtw_clustering}}\pysiglinewithargsret{\sphinxbfcode{\sphinxupquote{class }}\sphinxcode{\sphinxupquote{dtwhaclustering.dtw\_analysis.}}\sphinxbfcode{\sphinxupquote{dtw\_clustering}}}{\emph{\DUrole{n}{matrix}}, \emph{\DUrole{n}{labels}\DUrole{o}{=}\DUrole{default_value}{{[}{]}}}, \emph{\DUrole{n}{longitudes}\DUrole{o}{=}\DUrole{default_value}{{[}{]}}}, \emph{\DUrole{n}{latitudes}\DUrole{o}{=}\DUrole{default_value}{{[}{]}}}}{}
\sphinxAtStartPar
Bases: \sphinxcode{\sphinxupquote{object}}
\index{compute\_cluster() (dtwhaclustering.dtw\_analysis.dtw\_clustering method)@\spxentry{compute\_cluster()}\spxextra{dtwhaclustering.dtw\_analysis.dtw\_clustering method}}

\begin{fulllineitems}
\phantomsection\label{\detokenize{modules/dtw_analysis:dtwhaclustering.dtw_analysis.dtw_clustering.compute_cluster}}\pysiglinewithargsret{\sphinxbfcode{\sphinxupquote{compute\_cluster}}}{\emph{\DUrole{n}{clusterMatrix}\DUrole{o}{=}\DUrole{default_value}{None}}, \emph{\DUrole{n}{window}\DUrole{o}{=}\DUrole{default_value}{\textquotesingle{}constrained\textquotesingle{}}}, \emph{\DUrole{n}{windowfrac}\DUrole{o}{=}\DUrole{default_value}{0.25}}}{}~\begin{quote}\begin{description}
\item[{Parameters}] \leavevmode
\sphinxAtStartPar
\sphinxstyleliteralstrong{\sphinxupquote{compute\_cluster}} \textendash{} data matrix to cluster

\item[{Returns}] \leavevmode
\sphinxAtStartPar
model, cluster\_idx

\end{description}\end{quote}

\end{fulllineitems}

\index{compute\_cut\_off\_inconsistency() (dtwhaclustering.dtw\_analysis.dtw\_clustering method)@\spxentry{compute\_cut\_off\_inconsistency()}\spxextra{dtwhaclustering.dtw\_analysis.dtw\_clustering method}}

\begin{fulllineitems}
\phantomsection\label{\detokenize{modules/dtw_analysis:dtwhaclustering.dtw_analysis.dtw_clustering.compute_cut_off_inconsistency}}\pysiglinewithargsret{\sphinxbfcode{\sphinxupquote{compute\_cut\_off\_inconsistency}}}{\emph{\DUrole{n}{t}\DUrole{o}{=}\DUrole{default_value}{None}}, \emph{\DUrole{n}{depth}\DUrole{o}{=}\DUrole{default_value}{2}}, \emph{\DUrole{n}{criterion}\DUrole{o}{=}\DUrole{default_value}{\textquotesingle{}inconsistent\textquotesingle{}}}, \emph{\DUrole{n}{return\_cluster}\DUrole{o}{=}\DUrole{default_value}{False}}}{}
\sphinxAtStartPar
Calculate inconsistency statistics on a linkage matrix following \sphinxtitleref{scipy.cluster.hierarchy.inconsistent}. It compares each cluster merge’s height \sphinxtitleref{h} to the average \sphinxtitleref{avg} and normalize it by the standard deviation \sphinxtitleref{std} formed over the depth previous levels
\begin{quote}\begin{description}
\item[{Parameters}] \leavevmode\begin{itemize}
\item {} 
\sphinxAtStartPar
\sphinxstyleliteralstrong{\sphinxupquote{t}} (\sphinxstyleliteralemphasis{\sphinxupquote{scalar}}) \textendash{} threshold to apply when forming flat clusters. See scipy.cluster.hierarchy.fcluster for details

\item {} 
\sphinxAtStartPar
\sphinxstyleliteralstrong{\sphinxupquote{depth}} (\sphinxstyleliteralemphasis{\sphinxupquote{int}}) \textendash{} The maximum depth to perform the inconsistency calculation

\end{itemize}

\item[{Returns}] \leavevmode
\sphinxAtStartPar
maximum inconsistency coefficient for each non\sphinxhyphen{}singleton cluster and its children; the inconsistency matrix (matrix with rows of avg, std, count, inconsistency); cluster

\end{description}\end{quote}

\end{fulllineitems}

\index{compute\_dendrogram() (dtwhaclustering.dtw\_analysis.dtw\_clustering method)@\spxentry{compute\_dendrogram()}\spxextra{dtwhaclustering.dtw\_analysis.dtw\_clustering method}}

\begin{fulllineitems}
\phantomsection\label{\detokenize{modules/dtw_analysis:dtwhaclustering.dtw_analysis.dtw_clustering.compute_dendrogram}}\pysiglinewithargsret{\sphinxbfcode{\sphinxupquote{compute\_dendrogram}}}{\emph{\DUrole{n}{color\_thresh}\DUrole{o}{=}\DUrole{default_value}{None}}}{}
\end{fulllineitems}

\index{compute\_distance\_accl() (dtwhaclustering.dtw\_analysis.dtw\_clustering method)@\spxentry{compute\_distance\_accl()}\spxextra{dtwhaclustering.dtw\_analysis.dtw\_clustering method}}

\begin{fulllineitems}
\phantomsection\label{\detokenize{modules/dtw_analysis:dtwhaclustering.dtw_analysis.dtw_clustering.compute_distance_accl}}\pysiglinewithargsret{\sphinxbfcode{\sphinxupquote{compute\_distance\_accl}}}{\emph{\DUrole{n}{clusterMatrix}\DUrole{o}{=}\DUrole{default_value}{None}}}{}
\end{fulllineitems}

\index{compute\_distance\_matrix() (dtwhaclustering.dtw\_analysis.dtw\_clustering method)@\spxentry{compute\_distance\_matrix()}\spxextra{dtwhaclustering.dtw\_analysis.dtw\_clustering method}}

\begin{fulllineitems}
\phantomsection\label{\detokenize{modules/dtw_analysis:dtwhaclustering.dtw_analysis.dtw_clustering.compute_distance_matrix}}\pysiglinewithargsret{\sphinxbfcode{\sphinxupquote{compute\_distance\_matrix}}}{\emph{\DUrole{n}{compact}\DUrole{o}{=}\DUrole{default_value}{True}}, \emph{\DUrole{n}{block}\DUrole{o}{=}\DUrole{default_value}{None}}}{}
\end{fulllineitems}

\index{get\_linkage() (dtwhaclustering.dtw\_analysis.dtw\_clustering method)@\spxentry{get\_linkage()}\spxextra{dtwhaclustering.dtw\_analysis.dtw\_clustering method}}

\begin{fulllineitems}
\phantomsection\label{\detokenize{modules/dtw_analysis:dtwhaclustering.dtw_analysis.dtw_clustering.get_linkage}}\pysiglinewithargsret{\sphinxbfcode{\sphinxupquote{get\_linkage}}}{\emph{\DUrole{n}{clusterMatrix}\DUrole{o}{=}\DUrole{default_value}{None}}}{}
\end{fulllineitems}

\index{optimum\_cluster\_elbow() (dtwhaclustering.dtw\_analysis.dtw\_clustering method)@\spxentry{optimum\_cluster\_elbow()}\spxextra{dtwhaclustering.dtw\_analysis.dtw\_clustering method}}

\begin{fulllineitems}
\phantomsection\label{\detokenize{modules/dtw_analysis:dtwhaclustering.dtw_analysis.dtw_clustering.optimum_cluster_elbow}}\pysiglinewithargsret{\sphinxbfcode{\sphinxupquote{optimum\_cluster\_elbow}}}{\emph{\DUrole{n}{minmax}\DUrole{o}{=}\DUrole{default_value}{False}}, \emph{\DUrole{n}{plotloc}\DUrole{o}{=}\DUrole{default_value}{False}}}{}
\sphinxAtStartPar
Gives the optimum number of clusters required to express the maximum difference in the similarity using the elbow method

\end{fulllineitems}

\index{plot\_cluster\_geomap() (dtwhaclustering.dtw\_analysis.dtw\_clustering method)@\spxentry{plot\_cluster\_geomap()}\spxextra{dtwhaclustering.dtw\_analysis.dtw\_clustering method}}

\begin{fulllineitems}
\phantomsection\label{\detokenize{modules/dtw_analysis:dtwhaclustering.dtw_analysis.dtw_clustering.plot_cluster_geomap}}\pysiglinewithargsret{\sphinxbfcode{\sphinxupquote{plot\_cluster\_geomap}}}{\emph{\DUrole{n}{dtw\_distance}\DUrole{o}{=}\DUrole{default_value}{\textquotesingle{}optimal\textquotesingle{}}}, \emph{\DUrole{n}{minlon}\DUrole{o}{=}\DUrole{default_value}{None}}, \emph{\DUrole{n}{maxlon}\DUrole{o}{=}\DUrole{default_value}{None}}, \emph{\DUrole{n}{minlat}\DUrole{o}{=}\DUrole{default_value}{None}}, \emph{\DUrole{n}{maxlat}\DUrole{o}{=}\DUrole{default_value}{None}}, \emph{\DUrole{n}{figname}\DUrole{o}{=}\DUrole{default_value}{\textquotesingle{}dtw\_cluster.pdf\textquotesingle{}}}, \emph{\DUrole{n}{colorbar}\DUrole{o}{=}\DUrole{default_value}{True}}, \emph{\DUrole{n}{colorbarstep}\DUrole{o}{=}\DUrole{default_value}{1}}, \emph{\DUrole{n}{doffset}\DUrole{o}{=}\DUrole{default_value}{1}}, \emph{\DUrole{n}{dpi}\DUrole{o}{=}\DUrole{default_value}{720}}, \emph{\DUrole{n}{topo\_data}\DUrole{o}{=}\DUrole{default_value}{\textquotesingle{}@earth\_relief\_15s\textquotesingle{}}}, \emph{\DUrole{n}{plot\_topo}\DUrole{o}{=}\DUrole{default_value}{False}}, \emph{\DUrole{n}{markerstyle}\DUrole{o}{=}\DUrole{default_value}{\textquotesingle{}c0.3c\textquotesingle{}}}, \emph{\DUrole{n}{cmap\_topo}\DUrole{o}{=}\DUrole{default_value}{\textquotesingle{}topo\textquotesingle{}}}, \emph{\DUrole{n}{cmap\_data}\DUrole{o}{=}\DUrole{default_value}{\textquotesingle{}jet\textquotesingle{}}}, \emph{\DUrole{n}{projection}\DUrole{o}{=}\DUrole{default_value}{\textquotesingle{}M4i\textquotesingle{}}}, \emph{\DUrole{n}{topo\_cpt\_range}\DUrole{o}{=}\DUrole{default_value}{\textquotesingle{}\sphinxhyphen{}8000/8000/1000\textquotesingle{}}}, \emph{\DUrole{n}{landcolor}\DUrole{o}{=}\DUrole{default_value}{\textquotesingle{}\#666666\textquotesingle{}}}}{}
\sphinxAtStartPar
Plot the cluster points on a geographical map
\begin{quote}\begin{description}
\item[{Parameters}] \leavevmode\begin{itemize}
\item {} 
\sphinxAtStartPar
\sphinxstyleliteralstrong{\sphinxupquote{figname}} (\sphinxstyleliteralemphasis{\sphinxupquote{str}}) \textendash{} output figure name

\item {} 
\sphinxAtStartPar
\sphinxstyleliteralstrong{\sphinxupquote{colorbar}} (\sphinxstyleliteralemphasis{\sphinxupquote{bool}}) \textendash{} plot colorbar

\item {} 
\sphinxAtStartPar
\sphinxstyleliteralstrong{\sphinxupquote{colorbarstep}} (\sphinxstyleliteralemphasis{\sphinxupquote{int}}) \textendash{} step for the colorbar

\item {} 
\sphinxAtStartPar
\sphinxstyleliteralstrong{\sphinxupquote{cmap}} (\sphinxstyleliteralemphasis{\sphinxupquote{str}}) \textendash{} colormap for the cluster points

\item {} 
\sphinxAtStartPar
\sphinxstyleliteralstrong{\sphinxupquote{projection}} (\sphinxstyleliteralemphasis{\sphinxupquote{str}}) \textendash{} map projection

\item {} 
\sphinxAtStartPar
\sphinxstyleliteralstrong{\sphinxupquote{topo\_data}} (\sphinxstyleliteralemphasis{\sphinxupquote{str}}) \textendash{} topographic data resolution, see \sphinxhref{https://docs.generic-mapping-tools.org/latest/datasets/remote-data.html\#global-earth-relief-grids}{pygmt docs} for details

\item {} 
\sphinxAtStartPar
\sphinxstyleliteralstrong{\sphinxupquote{topo\_cpt\_range}} \textendash{} min/max/step for topographic color

\item {} 
\sphinxAtStartPar
\sphinxstyleliteralstrong{\sphinxupquote{landcolor}} (\sphinxstyleliteralemphasis{\sphinxupquote{str}}) \textendash{} color for land region

\item {} 
\sphinxAtStartPar
\sphinxstyleliteralstrong{\sphinxupquote{dpi}} (\sphinxstyleliteralemphasis{\sphinxupquote{int}}) \textendash{} output figure resolution

\end{itemize}

\end{description}\end{quote}

\end{fulllineitems}

\index{plot\_cluster\_geomap\_interpolated() (dtwhaclustering.dtw\_analysis.dtw\_clustering method)@\spxentry{plot\_cluster\_geomap\_interpolated()}\spxextra{dtwhaclustering.dtw\_analysis.dtw\_clustering method}}

\begin{fulllineitems}
\phantomsection\label{\detokenize{modules/dtw_analysis:dtwhaclustering.dtw_analysis.dtw_clustering.plot_cluster_geomap_interpolated}}\pysiglinewithargsret{\sphinxbfcode{\sphinxupquote{plot\_cluster\_geomap\_interpolated}}}{\emph{\DUrole{n}{dtw\_distance}\DUrole{o}{=}\DUrole{default_value}{\textquotesingle{}optimal\textquotesingle{}}}, \emph{\DUrole{n}{lonrange}\DUrole{o}{=}\DUrole{default_value}{(120.0, 122.0)}}, \emph{\DUrole{n}{latrange}\DUrole{o}{=}\DUrole{default_value}{(21.8, 25.6)}}, \emph{\DUrole{n}{gridstep}\DUrole{o}{=}\DUrole{default_value}{0.01}}, \emph{\DUrole{n}{figname}\DUrole{o}{=}\DUrole{default_value}{\textquotesingle{}dtw\_cluster\_interp.pdf\textquotesingle{}}}, \emph{\DUrole{n}{minlon}\DUrole{o}{=}\DUrole{default_value}{None}}, \emph{\DUrole{n}{maxlon}\DUrole{o}{=}\DUrole{default_value}{None}}, \emph{\DUrole{n}{minlat}\DUrole{o}{=}\DUrole{default_value}{None}}, \emph{\DUrole{n}{maxlat}\DUrole{o}{=}\DUrole{default_value}{None}}, \emph{\DUrole{n}{markerstyle}\DUrole{o}{=}\DUrole{default_value}{\textquotesingle{}c0.3c\textquotesingle{}}}, \emph{\DUrole{n}{dpi}\DUrole{o}{=}\DUrole{default_value}{720}}, \emph{\DUrole{n}{doffset}\DUrole{o}{=}\DUrole{default_value}{1}}, \emph{\DUrole{n}{plot\_data}\DUrole{o}{=}\DUrole{default_value}{True}}, \emph{\DUrole{n}{plot\_intrp}\DUrole{o}{=}\DUrole{default_value}{True}}}{}~\begin{quote}\begin{description}
\item[{Parameters}] \leavevmode\begin{itemize}
\item {} 
\sphinxAtStartPar
\sphinxstyleliteralstrong{\sphinxupquote{dtw\_distance}} (\sphinxstyleliteralemphasis{\sphinxupquote{str}}\sphinxstyleliteralemphasis{\sphinxupquote{ or }}\sphinxstyleliteralemphasis{\sphinxupquote{float}}) \textendash{} use \sphinxtitleref{dtw\_distance} value to obtain the cluster division. If \sphinxtitleref{optimal} then the optimal \sphinxtitleref{dtw\_distance} will be calculated

\item {} 
\sphinxAtStartPar
\sphinxstyleliteralstrong{\sphinxupquote{lonrange}} (\sphinxstyleliteralemphasis{\sphinxupquote{tuple}}) \textendash{} minimum and maximum of longitude values for interpolation

\item {} 
\sphinxAtStartPar
\sphinxstyleliteralstrong{\sphinxupquote{latrange}} (\sphinxstyleliteralemphasis{\sphinxupquote{tuple}}) \textendash{} minimum and maximum of latitude values for interpolation

\item {} 
\sphinxAtStartPar
\sphinxstyleliteralstrong{\sphinxupquote{gridstep}} (\sphinxstyleliteralemphasis{\sphinxupquote{float}}) \textendash{} step size for interpolation

\end{itemize}

\end{description}\end{quote}

\end{fulllineitems}

\index{plot\_cluster\_xymap() (dtwhaclustering.dtw\_analysis.dtw\_clustering method)@\spxentry{plot\_cluster\_xymap()}\spxextra{dtwhaclustering.dtw\_analysis.dtw\_clustering method}}

\begin{fulllineitems}
\phantomsection\label{\detokenize{modules/dtw_analysis:dtwhaclustering.dtw_analysis.dtw_clustering.plot_cluster_xymap}}\pysiglinewithargsret{\sphinxbfcode{\sphinxupquote{plot\_cluster\_xymap}}}{\emph{\DUrole{n}{dtw\_distance}\DUrole{o}{=}\DUrole{default_value}{\textquotesingle{}optimal\textquotesingle{}}}, \emph{\DUrole{n}{figname}\DUrole{o}{=}\DUrole{default_value}{None}}, \emph{\DUrole{n}{xlabel}\DUrole{o}{=}\DUrole{default_value}{\textquotesingle{}x\textquotesingle{}}}, \emph{\DUrole{n}{ylabel}\DUrole{o}{=}\DUrole{default_value}{\textquotesingle{}y\textquotesingle{}}}, \emph{\DUrole{n}{colorbar}\DUrole{o}{=}\DUrole{default_value}{True}}, \emph{\DUrole{n}{colorbarstep}\DUrole{o}{=}\DUrole{default_value}{1}}, \emph{\DUrole{n}{scale}\DUrole{o}{=}\DUrole{default_value}{2}}, \emph{\DUrole{n}{fontsize}\DUrole{o}{=}\DUrole{default_value}{26}}, \emph{\DUrole{n}{markersize}\DUrole{o}{=}\DUrole{default_value}{12}}, \emph{\DUrole{n}{axesfontsize}\DUrole{o}{=}\DUrole{default_value}{20}}, \emph{\DUrole{n}{xtickstep}\DUrole{o}{=}\DUrole{default_value}{1}}, \emph{\DUrole{n}{tickfontsize}\DUrole{o}{=}\DUrole{default_value}{20}}, \emph{\DUrole{n}{edgecolors}\DUrole{o}{=}\DUrole{default_value}{\textquotesingle{}k\textquotesingle{}}}, \emph{\DUrole{n}{cmap}\DUrole{o}{=}\DUrole{default_value}{\textquotesingle{}jet\textquotesingle{}}}, \emph{\DUrole{n}{linewidths}\DUrole{o}{=}\DUrole{default_value}{2}}, \emph{\DUrole{n}{cbarsize}\DUrole{o}{=}\DUrole{default_value}{18}}}{}
\sphinxAtStartPar
Plot the cluster points in a rectangular coordinate system
\begin{quote}\begin{description}
\item[{Parameters}] \leavevmode\begin{itemize}
\item {} 
\sphinxAtStartPar
\sphinxstyleliteralstrong{\sphinxupquote{dtw\_distance}} (\sphinxstyleliteralemphasis{\sphinxupquote{str}}\sphinxstyleliteralemphasis{\sphinxupquote{ or }}\sphinxstyleliteralemphasis{\sphinxupquote{float}}) \textendash{} use \sphinxtitleref{dtw\_distance} value to obtain the cluster division. If \sphinxtitleref{optimal} then the optimal \sphinxtitleref{dtw\_distance} will be calculated

\item {} 
\sphinxAtStartPar
\sphinxstyleliteralstrong{\sphinxupquote{figname}} (\sphinxstyleliteralemphasis{\sphinxupquote{str}}) \textendash{} output figure name

\item {} 
\sphinxAtStartPar
\sphinxstyleliteralstrong{\sphinxupquote{colorbar}} (\sphinxstyleliteralemphasis{\sphinxupquote{bool}}) \textendash{} plot colorbar

\item {} 
\sphinxAtStartPar
\sphinxstyleliteralstrong{\sphinxupquote{colorbarstep}} (\sphinxstyleliteralemphasis{\sphinxupquote{int}}) \textendash{} step for the colorbar

\item {} 
\sphinxAtStartPar
\sphinxstyleliteralstrong{\sphinxupquote{scale}} (\sphinxstyleliteralemphasis{\sphinxupquote{int}}) \textendash{} figure scale

\item {} 
\sphinxAtStartPar
\sphinxstyleliteralstrong{\sphinxupquote{markersize}} \textendash{} cluster points size

\item {} 
\sphinxAtStartPar
\sphinxstyleliteralstrong{\sphinxupquote{edgecolors}} \textendash{} marker edge color

\item {} 
\sphinxAtStartPar
\sphinxstyleliteralstrong{\sphinxupquote{cmap}} (\sphinxstyleliteralemphasis{\sphinxupquote{str}}) \textendash{} colormap for the cluster points

\end{itemize}

\item[{Returns}] \leavevmode
\sphinxAtStartPar
figure, axes

\end{description}\end{quote}

\end{fulllineitems}

\index{plot\_dendrogram() (dtwhaclustering.dtw\_analysis.dtw\_clustering method)@\spxentry{plot\_dendrogram()}\spxextra{dtwhaclustering.dtw\_analysis.dtw\_clustering method}}

\begin{fulllineitems}
\phantomsection\label{\detokenize{modules/dtw_analysis:dtwhaclustering.dtw_analysis.dtw_clustering.plot_dendrogram}}\pysiglinewithargsret{\sphinxbfcode{\sphinxupquote{plot\_dendrogram}}}{\emph{\DUrole{n}{figname}\DUrole{o}{=}\DUrole{default_value}{None}}, \emph{\DUrole{n}{figsize}\DUrole{o}{=}\DUrole{default_value}{(20, 8)}}, \emph{\DUrole{n}{xtickfontsize}\DUrole{o}{=}\DUrole{default_value}{26}}, \emph{\DUrole{n}{labelfontsize}\DUrole{o}{=}\DUrole{default_value}{26}}, \emph{\DUrole{n}{xlabel}\DUrole{o}{=}\DUrole{default_value}{\textquotesingle{}3\sphinxhyphen{}D Stations\textquotesingle{}}}, \emph{\DUrole{n}{ylabel}\DUrole{o}{=}\DUrole{default_value}{\textquotesingle{}DTW Distance\textquotesingle{}}}, \emph{\DUrole{n}{truncate\_p}\DUrole{o}{=}\DUrole{default_value}{None}}, \emph{\DUrole{n}{distance\_threshold}\DUrole{o}{=}\DUrole{default_value}{None}}, \emph{\DUrole{n}{annotate\_above}\DUrole{o}{=}\DUrole{default_value}{0}}, \emph{\DUrole{n}{plotpdf}\DUrole{o}{=}\DUrole{default_value}{True}}, \emph{\DUrole{n}{leaf\_rotation}\DUrole{o}{=}\DUrole{default_value}{0}}}{}~\begin{quote}\begin{description}
\item[{Parameters}] \leavevmode
\sphinxAtStartPar
\sphinxstyleliteralstrong{\sphinxupquote{truncate\_p}} \textendash{} show only last truncate\_p out of all merged branches

\end{description}\end{quote}

\end{fulllineitems}

\index{plot\_hac\_iteration() (dtwhaclustering.dtw\_analysis.dtw\_clustering method)@\spxentry{plot\_hac\_iteration()}\spxextra{dtwhaclustering.dtw\_analysis.dtw\_clustering method}}

\begin{fulllineitems}
\phantomsection\label{\detokenize{modules/dtw_analysis:dtwhaclustering.dtw_analysis.dtw_clustering.plot_hac_iteration}}\pysiglinewithargsret{\sphinxbfcode{\sphinxupquote{plot\_hac\_iteration}}}{\emph{\DUrole{n}{figname}\DUrole{o}{=}\DUrole{default_value}{None}}, \emph{\DUrole{n}{figsize}\DUrole{o}{=}\DUrole{default_value}{(10, 8)}}, \emph{\DUrole{n}{xtickfontsize}\DUrole{o}{=}\DUrole{default_value}{26}}, \emph{\DUrole{n}{labelfontsize}\DUrole{o}{=}\DUrole{default_value}{26}}, \emph{\DUrole{n}{xlabel}\DUrole{o}{=}\DUrole{default_value}{\textquotesingle{}Iteration \#\textquotesingle{}}}, \emph{\DUrole{n}{ylabel}\DUrole{o}{=}\DUrole{default_value}{\textquotesingle{}DTW Distance\textquotesingle{}}}, \emph{\DUrole{n}{plot\_color}\DUrole{o}{=}\DUrole{default_value}{\textquotesingle{}C0\textquotesingle{}}}, \emph{\DUrole{n}{plotpdf}\DUrole{o}{=}\DUrole{default_value}{True}}}{}
\end{fulllineitems}

\index{plot\_optimum\_cluster() (dtwhaclustering.dtw\_analysis.dtw\_clustering method)@\spxentry{plot\_optimum\_cluster()}\spxextra{dtwhaclustering.dtw\_analysis.dtw\_clustering method}}

\begin{fulllineitems}
\phantomsection\label{\detokenize{modules/dtw_analysis:dtwhaclustering.dtw_analysis.dtw_clustering.plot_optimum_cluster}}\pysiglinewithargsret{\sphinxbfcode{\sphinxupquote{plot\_optimum\_cluster}}}{\emph{\DUrole{n}{max\_d}\DUrole{o}{=}\DUrole{default_value}{None}}, \emph{\DUrole{n}{figname}\DUrole{o}{=}\DUrole{default_value}{None}}, \emph{\DUrole{n}{figsize}\DUrole{o}{=}\DUrole{default_value}{(10, 6)}}, \emph{\DUrole{n}{plotpdf}\DUrole{o}{=}\DUrole{default_value}{True}}, \emph{\DUrole{n}{xlabel}\DUrole{o}{=}\DUrole{default_value}{\textquotesingle{}Number of Clusters\textquotesingle{}}}, \emph{\DUrole{n}{ylabel}\DUrole{o}{=}\DUrole{default_value}{\textquotesingle{}Distance\textquotesingle{}}}, \emph{\DUrole{n}{accl\_label}\DUrole{o}{=}\DUrole{default_value}{\textquotesingle{}Curvature\textquotesingle{}}}, \emph{\DUrole{n}{accl\_color}\DUrole{o}{=}\DUrole{default_value}{\textquotesingle{}C10\textquotesingle{}}}, \emph{\DUrole{n}{dist\_label}\DUrole{o}{=}\DUrole{default_value}{\textquotesingle{}DTW Distance\textquotesingle{}}}, \emph{\DUrole{n}{dist\_color}\DUrole{o}{=}\DUrole{default_value}{\textquotesingle{}k\textquotesingle{}}}, \emph{\DUrole{n}{xlim}\DUrole{o}{=}\DUrole{default_value}{None}}, \emph{\DUrole{n}{legend\_outside}\DUrole{o}{=}\DUrole{default_value}{True}}, \emph{\DUrole{n}{fontsize}\DUrole{o}{=}\DUrole{default_value}{26}}, \emph{\DUrole{n}{xlabelfont}\DUrole{o}{=}\DUrole{default_value}{26}}, \emph{\DUrole{n}{ylabelfont}\DUrole{o}{=}\DUrole{default_value}{26}}}{}~\begin{quote}\begin{description}
\item[{Parameters}] \leavevmode
\sphinxAtStartPar
\sphinxstyleliteralstrong{\sphinxupquote{xlim}} (\sphinxstyleliteralemphasis{\sphinxupquote{list}}) \textendash{} x limits of the plot e.g., {[}1,2{]}

\end{description}\end{quote}

\end{fulllineitems}

\index{plot\_polar\_dendrogram() (dtwhaclustering.dtw\_analysis.dtw\_clustering method)@\spxentry{plot\_polar\_dendrogram()}\spxextra{dtwhaclustering.dtw\_analysis.dtw\_clustering method}}

\begin{fulllineitems}
\phantomsection\label{\detokenize{modules/dtw_analysis:dtwhaclustering.dtw_analysis.dtw_clustering.plot_polar_dendrogram}}\pysiglinewithargsret{\sphinxbfcode{\sphinxupquote{plot\_polar\_dendrogram}}}{\emph{\DUrole{n}{figsize}\DUrole{o}{=}\DUrole{default_value}{(20, 20)}}, \emph{\DUrole{n}{normfactor}\DUrole{o}{=}\DUrole{default_value}{None}}, \emph{\DUrole{n}{Nyticks}\DUrole{o}{=}\DUrole{default_value}{7}}, \emph{\DUrole{n}{gap}\DUrole{o}{=}\DUrole{default_value}{0.05}}, \emph{\DUrole{n}{Nsmooth}\DUrole{o}{=}\DUrole{default_value}{100}}, \emph{\DUrole{n}{linewidth}\DUrole{o}{=}\DUrole{default_value}{1}}, \emph{\DUrole{n}{xtickfontsize}\DUrole{o}{=}\DUrole{default_value}{20}}, \emph{\DUrole{n}{ytickfontsize}\DUrole{o}{=}\DUrole{default_value}{20}}, \emph{\DUrole{n}{plotstyle}\DUrole{o}{=}\DUrole{default_value}{\textquotesingle{}seaborn\textquotesingle{}}}, \emph{\DUrole{n}{figname}\DUrole{o}{=}\DUrole{default_value}{None}}, \emph{\DUrole{n}{plotpdf}\DUrole{o}{=}\DUrole{default_value}{True}}, \emph{\DUrole{n}{gridcolor}\DUrole{o}{=}\DUrole{default_value}{None}}, \emph{\DUrole{n}{gridstyle}\DUrole{o}{=}\DUrole{default_value}{\textquotesingle{}\sphinxhyphen{}\sphinxhyphen{}\textquotesingle{}}}, \emph{\DUrole{n}{gridwidth}\DUrole{o}{=}\DUrole{default_value}{1}}, \emph{\DUrole{n}{tickfontweight}\DUrole{o}{=}\DUrole{default_value}{\textquotesingle{}bold\textquotesingle{}}}, \emph{\DUrole{n}{distance\_threshold}\DUrole{o}{=}\DUrole{default_value}{None}}}{}
\sphinxAtStartPar
Plot polar dendrogram of the clustering result
\begin{quote}\begin{description}
\item[{Parameters}] \leavevmode\begin{itemize}
\item {} 
\sphinxAtStartPar
\sphinxstyleliteralstrong{\sphinxupquote{figsize}} \textendash{} figure size

\item {} 
\sphinxAtStartPar
\sphinxstyleliteralstrong{\sphinxupquote{normfactor}} \textendash{} (optional) normalization factor for the log spacing between the yticks, \sphinxhyphen{}np.log(dcoord+normfactor)

\item {} 
\sphinxAtStartPar
\sphinxstyleliteralstrong{\sphinxupquote{Nyticks}} \textendash{} number of y ticks

\item {} 
\sphinxAtStartPar
\sphinxstyleliteralstrong{\sphinxupquote{gap}} \textendash{} gap for the yticks

\item {} 
\sphinxAtStartPar
\sphinxstyleliteralstrong{\sphinxupquote{plotstyle}} \textendash{} matplotlib plot style, \sphinxtitleref{plotstyle} by default

\item {} 
\sphinxAtStartPar
\sphinxstyleliteralstrong{\sphinxupquote{distance\_threshold}} \textendash{} str, float. threshold for the coloring of branches. If str=”optimal”, then the optimal number of clusters based on elbow method will be used

\end{itemize}

\end{description}\end{quote}

\end{fulllineitems}

\index{plot\_signals() (dtwhaclustering.dtw\_analysis.dtw\_clustering method)@\spxentry{plot\_signals()}\spxextra{dtwhaclustering.dtw\_analysis.dtw\_clustering method}}

\begin{fulllineitems}
\phantomsection\label{\detokenize{modules/dtw_analysis:dtwhaclustering.dtw_analysis.dtw_clustering.plot_signals}}\pysiglinewithargsret{\sphinxbfcode{\sphinxupquote{plot\_signals}}}{\emph{\DUrole{n}{figname}\DUrole{o}{=}\DUrole{default_value}{None}}, \emph{\DUrole{n}{figsize}\DUrole{o}{=}\DUrole{default_value}{(10, 6)}}, \emph{\DUrole{n}{fontsize}\DUrole{o}{=}\DUrole{default_value}{26}}}{}~\begin{quote}\begin{description}
\item[{Parameters}] \leavevmode\begin{itemize}
\item {} 
\sphinxAtStartPar
\sphinxstyleliteralstrong{\sphinxupquote{figname}} (\sphinxstyleliteralemphasis{\sphinxupquote{str}}) \textendash{} output figure name

\item {} 
\sphinxAtStartPar
\sphinxstyleliteralstrong{\sphinxupquote{figsize}} (\sphinxstyleliteralemphasis{\sphinxupquote{tuple}}) \textendash{} output figure size

\end{itemize}

\end{description}\end{quote}

\end{fulllineitems}

\index{significance\_test() (dtwhaclustering.dtw\_analysis.dtw\_clustering method)@\spxentry{significance\_test()}\spxextra{dtwhaclustering.dtw\_analysis.dtw\_clustering method}}

\begin{fulllineitems}
\phantomsection\label{\detokenize{modules/dtw_analysis:dtwhaclustering.dtw_analysis.dtw_clustering.significance_test}}\pysiglinewithargsret{\sphinxbfcode{\sphinxupquote{significance\_test}}}{\emph{\DUrole{n}{numsimulations}\DUrole{o}{=}\DUrole{default_value}{10}}, \emph{\DUrole{n}{outfile}\DUrole{o}{=}\DUrole{default_value}{\textquotesingle{}pickleFiles/dU\_accl\_sim\_results.pickle\textquotesingle{}}}, \emph{\DUrole{n}{fresh\_start}\DUrole{o}{=}\DUrole{default_value}{False}}}{}
\end{fulllineitems}


\end{fulllineitems}

\index{dtw\_signal\_pairs (class in dtwhaclustering.dtw\_analysis)@\spxentry{dtw\_signal\_pairs}\spxextra{class in dtwhaclustering.dtw\_analysis}}

\begin{fulllineitems}
\phantomsection\label{\detokenize{modules/dtw_analysis:dtwhaclustering.dtw_analysis.dtw_signal_pairs}}\pysiglinewithargsret{\sphinxbfcode{\sphinxupquote{class }}\sphinxcode{\sphinxupquote{dtwhaclustering.dtw\_analysis.}}\sphinxbfcode{\sphinxupquote{dtw\_signal\_pairs}}}{\emph{\DUrole{n}{s1}}, \emph{\DUrole{n}{s2}}, \emph{\DUrole{n}{labels}\DUrole{o}{=}\DUrole{default_value}{{[}\textquotesingle{}s1\textquotesingle{}, \textquotesingle{}s2\textquotesingle{}{]}}}}{}
\sphinxAtStartPar
Bases: \sphinxcode{\sphinxupquote{object}}
\index{compute\_distance() (dtwhaclustering.dtw\_analysis.dtw\_signal\_pairs method)@\spxentry{compute\_distance()}\spxextra{dtwhaclustering.dtw\_analysis.dtw\_signal\_pairs method}}

\begin{fulllineitems}
\phantomsection\label{\detokenize{modules/dtw_analysis:dtwhaclustering.dtw_analysis.dtw_signal_pairs.compute_distance}}\pysiglinewithargsret{\sphinxbfcode{\sphinxupquote{compute\_distance}}}{\emph{\DUrole{n}{pruning}\DUrole{o}{=}\DUrole{default_value}{True}}, \emph{\DUrole{n}{best\_path}\DUrole{o}{=}\DUrole{default_value}{False}}}{}
\sphinxAtStartPar
Returns the DTW distance
\begin{quote}\begin{description}
\item[{Parameters}] \leavevmode
\sphinxAtStartPar
\sphinxstyleliteralstrong{\sphinxupquote{pruning}} (\sphinxstyleliteralemphasis{\sphinxupquote{boolean}}) \textendash{} prunes computations by setting max\_dist to the Euclidean upper bound

\end{description}\end{quote}

\end{fulllineitems}

\index{compute\_warping\_path() (dtwhaclustering.dtw\_analysis.dtw\_signal\_pairs method)@\spxentry{compute\_warping\_path()}\spxextra{dtwhaclustering.dtw\_analysis.dtw\_signal\_pairs method}}

\begin{fulllineitems}
\phantomsection\label{\detokenize{modules/dtw_analysis:dtwhaclustering.dtw_analysis.dtw_signal_pairs.compute_warping_path}}\pysiglinewithargsret{\sphinxbfcode{\sphinxupquote{compute\_warping\_path}}}{\emph{\DUrole{n}{windowfrac}\DUrole{o}{=}\DUrole{default_value}{None}}, \emph{\DUrole{n}{psi}\DUrole{o}{=}\DUrole{default_value}{None}}, \emph{\DUrole{n}{fullmatrix}\DUrole{o}{=}\DUrole{default_value}{False}}}{}
\sphinxAtStartPar
Returns the DTW path
:param windowfrac: Fraction of the signal length. Only allow for shifts up to this amount away from the two diagonals.
:param psi: Up to psi number of start and end points of a sequence can be ignored if this would lead to a lower distance
:param full\_matrix: The full matrix of all warping paths (or accumulated cost matrix) is built

\end{fulllineitems}

\index{plot\_matrix() (dtwhaclustering.dtw\_analysis.dtw\_signal\_pairs method)@\spxentry{plot\_matrix()}\spxextra{dtwhaclustering.dtw\_analysis.dtw\_signal\_pairs method}}

\begin{fulllineitems}
\phantomsection\label{\detokenize{modules/dtw_analysis:dtwhaclustering.dtw_analysis.dtw_signal_pairs.plot_matrix}}\pysiglinewithargsret{\sphinxbfcode{\sphinxupquote{plot\_matrix}}}{\emph{\DUrole{n}{windowfrac}\DUrole{o}{=}\DUrole{default_value}{0.2}}, \emph{\DUrole{n}{psi}\DUrole{o}{=}\DUrole{default_value}{None}}, \emph{\DUrole{n}{figname}\DUrole{o}{=}\DUrole{default_value}{None}}, \emph{\DUrole{n}{shownumbers}\DUrole{o}{=}\DUrole{default_value}{False}}, \emph{\DUrole{n}{showlegend}\DUrole{o}{=}\DUrole{default_value}{True}}}{}
\sphinxAtStartPar
Plot the signals with the DTW matrix
\begin{quote}\begin{description}
\item[{Parameters}] \leavevmode
\sphinxAtStartPar
\sphinxstyleliteralstrong{\sphinxupquote{figname}} (\sphinxstyleliteralemphasis{\sphinxupquote{str}}) \textendash{} output figure name

\end{description}\end{quote}

\end{fulllineitems}

\index{plot\_signals() (dtwhaclustering.dtw\_analysis.dtw\_signal\_pairs method)@\spxentry{plot\_signals()}\spxextra{dtwhaclustering.dtw\_analysis.dtw\_signal\_pairs method}}

\begin{fulllineitems}
\phantomsection\label{\detokenize{modules/dtw_analysis:dtwhaclustering.dtw_analysis.dtw_signal_pairs.plot_signals}}\pysiglinewithargsret{\sphinxbfcode{\sphinxupquote{plot\_signals}}}{\emph{\DUrole{n}{figname}\DUrole{o}{=}\DUrole{default_value}{None}}, \emph{\DUrole{n}{figsize}\DUrole{o}{=}\DUrole{default_value}{(12, 6)}}}{}
\sphinxAtStartPar
Plot the signals
\begin{quote}\begin{description}
\item[{Parameters}] \leavevmode\begin{itemize}
\item {} 
\sphinxAtStartPar
\sphinxstyleliteralstrong{\sphinxupquote{figname}} (\sphinxstyleliteralemphasis{\sphinxupquote{str}}) \textendash{} output figure name

\item {} 
\sphinxAtStartPar
\sphinxstyleliteralstrong{\sphinxupquote{figsize}} (\sphinxstyleliteralemphasis{\sphinxupquote{tuple}}) \textendash{} output figure size

\end{itemize}

\item[{Returns}] \leavevmode
\sphinxAtStartPar
figure and axes object

\end{description}\end{quote}

\end{fulllineitems}

\index{plot\_warping\_path() (dtwhaclustering.dtw\_analysis.dtw\_signal\_pairs method)@\spxentry{plot\_warping\_path()}\spxextra{dtwhaclustering.dtw\_analysis.dtw\_signal\_pairs method}}

\begin{fulllineitems}
\phantomsection\label{\detokenize{modules/dtw_analysis:dtwhaclustering.dtw_analysis.dtw_signal_pairs.plot_warping_path}}\pysiglinewithargsret{\sphinxbfcode{\sphinxupquote{plot\_warping\_path}}}{\emph{\DUrole{n}{figname}\DUrole{o}{=}\DUrole{default_value}{None}}, \emph{\DUrole{n}{figsize}\DUrole{o}{=}\DUrole{default_value}{(12, 6)}}}{}
\sphinxAtStartPar
Plot the signals with the warping paths
\begin{quote}\begin{description}
\item[{Parameters}] \leavevmode\begin{itemize}
\item {} 
\sphinxAtStartPar
\sphinxstyleliteralstrong{\sphinxupquote{figname}} (\sphinxstyleliteralemphasis{\sphinxupquote{str}}) \textendash{} output figure name

\item {} 
\sphinxAtStartPar
\sphinxstyleliteralstrong{\sphinxupquote{figsize}} (\sphinxstyleliteralemphasis{\sphinxupquote{tuple}}) \textendash{} output figure size

\end{itemize}

\end{description}\end{quote}

\end{fulllineitems}


\end{fulllineitems}

\index{noise\_robustness\_test() (in module dtwhaclustering.dtw\_analysis)@\spxentry{noise\_robustness\_test()}\spxextra{in module dtwhaclustering.dtw\_analysis}}

\begin{fulllineitems}
\phantomsection\label{\detokenize{modules/dtw_analysis:dtwhaclustering.dtw_analysis.noise_robustness_test}}\pysiglinewithargsret{\sphinxcode{\sphinxupquote{dtwhaclustering.dtw\_analysis.}}\sphinxbfcode{\sphinxupquote{noise\_robustness\_test}}}{\emph{\DUrole{n}{clean\_df}}, \emph{\DUrole{n}{scale}\DUrole{o}{=}\DUrole{default_value}{0.1}}}{}
\end{fulllineitems}

\index{plot\_cluster() (in module dtwhaclustering.dtw\_analysis)@\spxentry{plot\_cluster()}\spxextra{in module dtwhaclustering.dtw\_analysis}}

\begin{fulllineitems}
\phantomsection\label{\detokenize{modules/dtw_analysis:dtwhaclustering.dtw_analysis.plot_cluster}}\pysiglinewithargsret{\sphinxcode{\sphinxupquote{dtwhaclustering.dtw\_analysis.}}\sphinxbfcode{\sphinxupquote{plot\_cluster}}}{\emph{\DUrole{n}{lons}}, \emph{\DUrole{n}{lats}}, \emph{\DUrole{n}{figname}\DUrole{o}{=}\DUrole{default_value}{None}}, \emph{\DUrole{n}{figsize}\DUrole{o}{=}\DUrole{default_value}{(10, 10)}}, \emph{\DUrole{n}{plotpdf}\DUrole{o}{=}\DUrole{default_value}{True}}, \emph{\DUrole{n}{labels}\DUrole{o}{=}\DUrole{default_value}{{[}{]}}}, \emph{\DUrole{n}{markersize}\DUrole{o}{=}\DUrole{default_value}{20}}}{}~\begin{quote}\begin{description}
\item[{Parameters}] \leavevmode\begin{itemize}
\item {} 
\sphinxAtStartPar
\sphinxstyleliteralstrong{\sphinxupquote{figname}} (\sphinxstyleliteralemphasis{\sphinxupquote{str}}) \textendash{} output figure name

\item {} 
\sphinxAtStartPar
\sphinxstyleliteralstrong{\sphinxupquote{figsize}} (\sphinxstyleliteralemphasis{\sphinxupquote{tuple}}) \textendash{} output figure size

\end{itemize}

\end{description}\end{quote}

\end{fulllineitems}

\index{plot\_signals() (in module dtwhaclustering.dtw\_analysis)@\spxentry{plot\_signals()}\spxextra{in module dtwhaclustering.dtw\_analysis}}

\begin{fulllineitems}
\phantomsection\label{\detokenize{modules/dtw_analysis:dtwhaclustering.dtw_analysis.plot_signals}}\pysiglinewithargsret{\sphinxcode{\sphinxupquote{dtwhaclustering.dtw\_analysis.}}\sphinxbfcode{\sphinxupquote{plot\_signals}}}{\emph{\DUrole{n}{matrix}}, \emph{\DUrole{n}{labels}\DUrole{o}{=}\DUrole{default_value}{{[}{]}}}, \emph{\DUrole{n}{figname}\DUrole{o}{=}\DUrole{default_value}{None}}, \emph{\DUrole{n}{plotpdf}\DUrole{o}{=}\DUrole{default_value}{True}}, \emph{\DUrole{n}{figsize}\DUrole{o}{=}\DUrole{default_value}{(10, 8)}}, \emph{\DUrole{n}{ylabelsize}\DUrole{o}{=}\DUrole{default_value}{26}}, \emph{\DUrole{n}{color}\DUrole{o}{=}\DUrole{default_value}{None}}}{}
\sphinxAtStartPar
Plot signals
\begin{quote}\begin{description}
\item[{Parameters}] \leavevmode\begin{itemize}
\item {} 
\sphinxAtStartPar
\sphinxstyleliteralstrong{\sphinxupquote{color}} \textendash{} list of colors. If None then matplotlib defaults color sequence will be used

\item {} 
\sphinxAtStartPar
\sphinxstyleliteralstrong{\sphinxupquote{figname}} (\sphinxstyleliteralemphasis{\sphinxupquote{str}}) \textendash{} output figure name

\item {} 
\sphinxAtStartPar
\sphinxstyleliteralstrong{\sphinxupquote{figsize}} (\sphinxstyleliteralemphasis{\sphinxupquote{tuple}}) \textendash{} output figure size

\end{itemize}

\end{description}\end{quote}

\end{fulllineitems}

\index{shuffle\_signals() (in module dtwhaclustering.dtw\_analysis)@\spxentry{shuffle\_signals()}\spxextra{in module dtwhaclustering.dtw\_analysis}}

\begin{fulllineitems}
\phantomsection\label{\detokenize{modules/dtw_analysis:dtwhaclustering.dtw_analysis.shuffle_signals}}\pysiglinewithargsret{\sphinxcode{\sphinxupquote{dtwhaclustering.dtw\_analysis.}}\sphinxbfcode{\sphinxupquote{shuffle\_signals}}}{\emph{\DUrole{n}{matrix}}, \emph{\DUrole{n}{labels}\DUrole{o}{=}\DUrole{default_value}{{[}{]}}}, \emph{\DUrole{n}{plot\_signals}\DUrole{o}{=}\DUrole{default_value}{False}}, \emph{\DUrole{n}{figsize}\DUrole{o}{=}\DUrole{default_value}{(10, 12)}}, \emph{\DUrole{n}{figname}\DUrole{o}{=}\DUrole{default_value}{None}}, \emph{\DUrole{n}{plotpdf}\DUrole{o}{=}\DUrole{default_value}{True}}}{}~\begin{quote}\begin{description}
\item[{Parameters}] \leavevmode\begin{itemize}
\item {} 
\sphinxAtStartPar
\sphinxstyleliteralstrong{\sphinxupquote{figname}} (\sphinxstyleliteralemphasis{\sphinxupquote{str}}) \textendash{} output figure name

\item {} 
\sphinxAtStartPar
\sphinxstyleliteralstrong{\sphinxupquote{figsize}} (\sphinxstyleliteralemphasis{\sphinxupquote{tuple}}) \textendash{} output figure size

\end{itemize}

\end{description}\end{quote}

\end{fulllineitems}



\chapter{dtwhaclustering.plot\_stations}
\label{\detokenize{modules/plot_stations:module-dtwhaclustering.plot_stations}}\label{\detokenize{modules/plot_stations:dtwhaclustering-plot-stations}}\label{\detokenize{modules/plot_stations::doc}}\index{module@\spxentry{module}!dtwhaclustering.plot\_stations@\spxentry{dtwhaclustering.plot\_stations}}\index{dtwhaclustering.plot\_stations@\spxentry{dtwhaclustering.plot\_stations}!module@\spxentry{module}}
\sphinxAtStartPar
Plot topographic station map
\begin{quote}\begin{description}
\item[{author}] \leavevmode
\sphinxAtStartPar
Utpal Kumar, Institute of Earth Sciences, Academia Sinica

\end{description}\end{quote}
\index{plot\_station\_map() (in module dtwhaclustering.plot\_stations)@\spxentry{plot\_station\_map()}\spxextra{in module dtwhaclustering.plot\_stations}}

\begin{fulllineitems}
\phantomsection\label{\detokenize{modules/plot_stations:dtwhaclustering.plot_stations.plot_station_map}}\pysiglinewithargsret{\sphinxcode{\sphinxupquote{dtwhaclustering.plot\_stations.}}\sphinxbfcode{\sphinxupquote{plot\_station\_map}}}{\emph{\DUrole{n}{station\_data}}, \emph{\DUrole{n}{minlon}\DUrole{o}{=}\DUrole{default_value}{None}}, \emph{\DUrole{n}{maxlon}\DUrole{o}{=}\DUrole{default_value}{None}}, \emph{\DUrole{n}{minlat}\DUrole{o}{=}\DUrole{default_value}{None}}, \emph{\DUrole{n}{maxlat}\DUrole{o}{=}\DUrole{default_value}{None}}, \emph{\DUrole{n}{outfig}\DUrole{o}{=}\DUrole{default_value}{\textquotesingle{}station\_map.png\textquotesingle{}}}, \emph{\DUrole{n}{datacolor}\DUrole{o}{=}\DUrole{default_value}{\textquotesingle{}blue\textquotesingle{}}}, \emph{\DUrole{n}{topo\_data}\DUrole{o}{=}\DUrole{default_value}{\textquotesingle{}@earth\_relief\_15s\textquotesingle{}}}, \emph{\DUrole{n}{cmap}\DUrole{o}{=}\DUrole{default_value}{\textquotesingle{}etopo1\textquotesingle{}}}, \emph{\DUrole{n}{projection}\DUrole{o}{=}\DUrole{default_value}{\textquotesingle{}M4i\textquotesingle{}}}, \emph{\DUrole{n}{datalabel}\DUrole{o}{=}\DUrole{default_value}{\textquotesingle{}Stations\textquotesingle{}}}, \emph{\DUrole{n}{markerstyle}\DUrole{o}{=}\DUrole{default_value}{\textquotesingle{}i10p\textquotesingle{}}}, \emph{\DUrole{n}{random\_station\_label}\DUrole{o}{=}\DUrole{default_value}{False}}, \emph{\DUrole{n}{stn\_labels}\DUrole{o}{=}\DUrole{default_value}{None}}, \emph{\DUrole{n}{justify}\DUrole{o}{=}\DUrole{default_value}{\textquotesingle{}left\textquotesingle{}}}, \emph{\DUrole{n}{labelfont}\DUrole{o}{=}\DUrole{default_value}{\textquotesingle{}6p,Helvetica\sphinxhyphen{}Bold,black\textquotesingle{}}}, \emph{\DUrole{n}{offset}\DUrole{o}{=}\DUrole{default_value}{\textquotesingle{}5p/\sphinxhyphen{}5p\textquotesingle{}}}, \emph{\DUrole{n}{stn\_labels\_color}\DUrole{o}{=}\DUrole{default_value}{\textquotesingle{}red\textquotesingle{}}}, \emph{\DUrole{n}{rand\_justify}\DUrole{o}{=}\DUrole{default_value}{False}}}{}
\sphinxAtStartPar
Plot topographic station map using PyGMT
\begin{quote}\begin{description}
\item[{Parameters}] \leavevmode\begin{itemize}
\item {} 
\sphinxAtStartPar
\sphinxstyleliteralstrong{\sphinxupquote{station\_data}} \textendash{} Pandas dataframe containing columns \sphinxtitleref{lon}, \sphinxtitleref{lat}

\item {} 
\sphinxAtStartPar
\sphinxstyleliteralstrong{\sphinxupquote{minlon}} \textendash{} Minimum longitude of the map (optional)

\item {} 
\sphinxAtStartPar
\sphinxstyleliteralstrong{\sphinxupquote{maxlon}} \textendash{} Maximum longitude of the map (optional)

\item {} 
\sphinxAtStartPar
\sphinxstyleliteralstrong{\sphinxupquote{minlat}} \textendash{} Minimum latitude of the map (optional)

\item {} 
\sphinxAtStartPar
\sphinxstyleliteralstrong{\sphinxupquote{maxlat}} \textendash{} Maximum latitude of the map (optional)

\item {} 
\sphinxAtStartPar
\sphinxstyleliteralstrong{\sphinxupquote{datacolor}} \textendash{} color of the data point to plot

\item {} 
\sphinxAtStartPar
\sphinxstyleliteralstrong{\sphinxupquote{topo\_data}} \textendash{} etopo data file

\item {} 
\sphinxAtStartPar
\sphinxstyleliteralstrong{\sphinxupquote{cmap}} \textendash{} colormap for the output topography

\item {} 
\sphinxAtStartPar
\sphinxstyleliteralstrong{\sphinxupquote{projection}} \textendash{} projection of the map. Mercator of width 4 inch by default

\item {} 
\sphinxAtStartPar
\sphinxstyleliteralstrong{\sphinxupquote{datalabel}} \textendash{} Label for the data

\item {} 
\sphinxAtStartPar
\sphinxstyleliteralstrong{\sphinxupquote{markerstyle}} \textendash{} Style of the marker. Inverted triangle of size 10p by default.

\item {} 
\sphinxAtStartPar
\sphinxstyleliteralstrong{\sphinxupquote{outfig}} \textendash{} Output figure path

\item {} 
\sphinxAtStartPar
\sphinxstyleliteralstrong{\sphinxupquote{random\_station\_label}} \textendash{} int. label randomly selected \sphinxtitleref{random\_station\_label} of stations

\end{itemize}

\end{description}\end{quote}

\begin{sphinxVerbatim}[commandchars=\\\{\}]
\PYG{k+kn}{from} \PYG{n+nn}{dtwhaclustering} \PYG{k+kn}{import} \PYG{n}{plot\PYGZus{}stations}
\PYG{n}{plot\PYGZus{}stations}\PYG{o}{.}\PYG{n}{plot\PYGZus{}station\PYGZus{}map}\PYG{p}{(}\PYG{n}{station\PYGZus{}data} \PYG{o}{=} \PYG{l+s+s1}{\PYGZsq{}}\PYG{l+s+s1}{helper\PYGZus{}files/selected\PYGZus{}stations\PYGZus{}info.txt}\PYG{l+s+s1}{\PYGZsq{}}\PYG{p}{,} \PYG{n}{outfig}\PYG{o}{=}\PYG{l+s+sa}{f}\PYG{l+s+s1}{\PYGZsq{}}\PYG{l+s+si}{\PYGZob{}}\PYG{n}{outloc}\PYG{l+s+si}{\PYGZcb{}}\PYG{l+s+s1}{/station\PYGZus{}map.pdf}\PYG{l+s+s1}{\PYGZsq{}}\PYG{p}{)}
\end{sphinxVerbatim}

\end{fulllineitems}



\chapter{Indices and tables}
\label{\detokenize{index:indices-and-tables}}\begin{itemize}
\item {} 
\sphinxAtStartPar
\DUrole{xref,std,std-ref}{genindex}

\item {} 
\sphinxAtStartPar
\DUrole{xref,std,std-ref}{modindex}

\item {} 
\sphinxAtStartPar
\DUrole{xref,std,std-ref}{search}

\end{itemize}


\renewcommand{\indexname}{Python Module Index}
\begin{sphinxtheindex}
\let\bigletter\sphinxstyleindexlettergroup
\bigletter{d}
\item\relax\sphinxstyleindexentry{dtwhaclustering}\sphinxstyleindexpageref{modules/module_contents:\detokenize{module-dtwhaclustering}}
\item\relax\sphinxstyleindexentry{dtwhaclustering.analysis\_support}\sphinxstyleindexpageref{modules/analysis_support:\detokenize{module-dtwhaclustering.analysis_support}}
\item\relax\sphinxstyleindexentry{dtwhaclustering.dtw\_analysis}\sphinxstyleindexpageref{modules/dtw_analysis:\detokenize{module-dtwhaclustering.dtw_analysis}}
\item\relax\sphinxstyleindexentry{dtwhaclustering.leastSquareModeling}\sphinxstyleindexpageref{modules/leastSquareModeling:\detokenize{module-dtwhaclustering.leastSquareModeling}}
\item\relax\sphinxstyleindexentry{dtwhaclustering.plot\_linear\_trend}\sphinxstyleindexpageref{modules/plot_linear_trend:\detokenize{module-dtwhaclustering.plot_linear_trend}}
\item\relax\sphinxstyleindexentry{dtwhaclustering.plot\_stations}\sphinxstyleindexpageref{modules/plot_stations:\detokenize{module-dtwhaclustering.plot_stations}}
\end{sphinxtheindex}

\renewcommand{\indexname}{Index}
\printindex
\end{document}